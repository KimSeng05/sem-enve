%=== PACKAGES ==================================================================
  \documentclass[a4paper]{article}
  %Appearance
  \usepackage{microtype} %better typography
  \usepackage[english]{babel}
  \usepackage{fancyhdr} %for fancy page style
  \usepackage{color} %Added support for color text

  %Document Dynamics
  \usepackage[colorlinks=true]{hyperref} %Manage links
  \hypersetup{linkcolor=blue}
  \usepackage{url}
  \usepackage{geometry} %For easy management of document margins
  \usepackage{graphicx} %Allows for insertion of images in document
  \graphicspath{ {figures/} {images/} }

  %Technical
  \usepackage{amsmath}
  \usepackage{amsthm}
  \usepackage{amssymb}
  \usepackage{tikz}
  \usepackage{listings} %To insert code blocks

  %Drafting
  \usepackage{todonotes}

%=== TITLE =====================================================================
  \title{\Large\textbf{BLDC Motor Controller Design}}
  \author{Khanh J. Phan}
  \date{Last Updated: \today}

%=== FRONT MATTER ==============================================================
  \begin{document}
  \pagestyle{fancy} \lhead{} \rhead{}
  \maketitle
  \tableofcontents
  %\listoffigures
  %\listoftables

%=== BODY ======================================================================
\section{Motor Parameters}
The motor selection is the first step taken in order to design the motor
controller. The motor was selected based on the application it was to be used
in. The vehicles average and maximum speeds, and peak acceleration/deceleration
was used as the criterion.
The motor selected was the \href{http://www.hobbyking.com/hobbyking/store/__14404__Turnigy_G46_Brushless_Outrunner_670kv_46_Glow_.html}{Turnigy
G46 Brushless Outrunner}.

\begin{figure}[h!]
    \centering
    \begin{tabular}{ll}
    \hline
    Parameter & Value \\
    \hline
    Recommended Battery & 4-5 Cells @ 14.4-18.5V\\
    RPM & 670kv\\
    Max Current & 40A\\
    No Load Current & 10V/3.9A\\
    Current Capacity & 55A/15sec\\
    Internal Resistance & 0.04 ohm\\
    Weight & 303g\\
    Diameter of Shaft & 6mm\\
    Dimensions & 76x50mm\\
    \hline
    \end{tabular}
    \caption{Motor Parameters}
\end{figure}

\subsection{Motor Demands}
The boundary parameters mentioned previously are also essential in selecting
the components for the power stage of the motor controller. These parameters are
as follows;

\begin{figure}[h!]
    \centering
    \begin{tabular}{ll}
    \hline
    Parameter & Value \\
    \hline
    Average Speed & 30 km h$^{-1}$\\
    Peak Speed & 52 km h$^{-1}$\\
    Wheel diameter & 0.6m\\
    Wheel Reduction Ratio & 1:10\\
    Average RPM (for switching frequency) & 2653 RPM\\
    Peak RPM (for switching frequency) & 4598 RPM \\
    Peak Acceleration (for dv/dt FET capabilities) & 0.856 RPM s$^{-1}$\\
    \hline
    \end{tabular}
    \caption{Motor Boundaries}
\end{figure}

%=== MOSFETS ===================================================================
\section{MOSFETs}
The selected MOSFET based on the motor parameters was the \href{http://www.electusdistribution.com.au/products_uploaded/ZT-2468.pdf}{IRF1405}

A summary of the MOSFETs parameters is below:

\begin{figure}[h!]
    \centering
    \footnotesize
    \begin{tabular}{llll}
    \hline
    & Parameter & Value & Conditions \\
    \hline
    $V_{(BR)DSS}$ & Drain-to-Source Breakdown Voltage & 55V & $V_{GS}=0V,~I_D=250\mu A$\\
    $I_{D}$ & Continuous Drain Current & 169A & $V_{GS}=10V,~T_C=25^{\circ}C$\\
    $I_{D}$ & Continuous Drain Current & 118A & $V_{GS}=10V,~T_C=100^{\circ}C$\\
    $I_{DM}$ & Pulsed Drain Current & 680A & \\
    $T_J$ & Operating Junction Temperature Range & $-55^{\circ}C - 175^{\circ}C$ & \\
    $Q_G$ & Total Gate Charge & $170nC - 260nC$ & $I_D=101A,~V_{DS}=44V,~V_{GS}=10V$\\
    $Q_{gs}$ & Gate-to-Source Charege & $44nC - 66nC$ & $I_D=101A,~V_{DS}=44V,~V_{GS}=10V$\\
    $Q_{gd}$ & Gate-to-Drain (Miller) Charge & $62nC - 93nC$ & $I_D=101A,~V_{DS}=44V,~V_{GS}=10V$\\
    $t_{r}$ & Rise time & 190ns & $V_{DD}=38V,~I_D=101A,~R_G=1.1\Omega,~V_{GS}=10V,$\\
    $t_{f}$ & Fall time & 110ns & $V_{DD}=38V,~I_D=101A,~R_G=1.1\Omega,~V_{GS}=10V,$\\
    \hline
    \end{tabular}
    \caption{IRF1405 Specifications}
\end{figure}

%=== DRIVER ====================================================================
\section{Gate Driver}
The N-Channel MOSFET gate driver is selected based on the MOSFET specifications.
Additional criteria such as functionality and minimal power dissipation is
secondary.

%-----------------------------------------------------------
\subsection{Peak Current Drive Requirements}
This is the primary criterion for the driver selection. This is primarly based
on how fast the application (motor) requires the MOSFET to be turned on and off
(rise and fall time of the gate voltage).

This turn on/off speed, $dT$, is in turn related to how fast the gate
capacitance of the MOSFET can be charged and discharged.
    \begin{equation}
    dT = \frac{\left[ dV \times C\right]}{I} = \frac{Q}{I}
    \end{equation}

Where;
    \begin{itemize}
    \item $dV$ = Gate voltage
    \item $C$ = Gate capacitance
    \item $I$ = Peak drive current
    \item $Q$ = Total gate charge
    \end{itemize}

In the case of the IRF1405, the total gate charge is \todo{not sure,
used max total gate charge} 260nC and a turn on/off time of $t = t_r + t_f =
300ns$

Thus we have a required average current draw of
    \begin{equation}
    I = \frac{Q}{dT} = \frac{260 \times 10^{-9}}{300 \times 10^{-9}} = 0.9A
    \end{equation}

%-----------------------------------------------------------
\subsection{Power Dissipation}
There are three elements due to the charging and discharging of the gate
capacitance of the MOSFET:
    \begin{enumerate}
    \item Power dissipation due to the charging and discharging of the gate
    capacitance of the MOSFET, $P_C$
    \item Power dissipation due to quiescent current draw of the MOSFET driver,
    $P_Q$
    \item Power dissipation due to cross-conduction (shoot-through) current in
    the MOSFET driver, $P_S$
    \end{enumerate}

The following equations apply:
    \begin{subequations}
        \begin{align}
        P_C = C_G\cdot V_{DD}^2\cdot F\\
        P_Q = \left( I_{QH}\cdot D + I_{QL}\cdot\left( I-D\right)\right)\cdot
        V_{DD}\\
        P_S = CC\cdot F\cdot V_{DD}\\
        \end{align}
    \end{subequations}

Where;
    \begin{itemize}
    \item $C_G$ is the MOSFET Gate Capacitance
    \item $V_DD$ is the supply voltage of the MOSFET driver
    \item $F$ is the switching frequency
    \item $I_{QH}$ is the Quiescent current of the driver with the input in the
    high state
    \item $D$ is the Duty cycle of the switching waveform
    \item $I_{QL}$ the quiescent current of the driver with the input in the low
    side
    \end{itemize}

%-----------------------------------------------------------
\subsection{IRS2181}
The \href{http://www.irf.com/product-info/datasheets/data/irs2181.pdf}{IRS2181}
was the selected MOSFET driver. It is fully operational up to 600V with 1.4-1.8A
source/sink current capabilities.

%=== MCU =======================================================================
\section{Microcontroller}

%=== REGULATION ================================================================
\section{Regulation}

%=== CURRENT SENSING ===========================================================
\section{Current Sensing}

\end{document}

