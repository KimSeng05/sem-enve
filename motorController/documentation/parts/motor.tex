\section{Motor Parameters}
The motor selection is the first step taken in order to design the motor
controller. The motor was selected based on the application it was to be used
in. The vehicles average and maximum speeds, and peak acceleration/deceleration
was used as the criterion.
The motor selected was the \href{http://www.hobbyking.com/hobbyking/store/__14404__Turnigy_G46_Brushless_Outrunner_670kv_46_Glow_.html}{Turnigy
G46 Brushless Outrunner}.

\begin{figure}[h!]
        \centering
        \begin{tabular}{ll}
        \hline
        Parameter & Value \\
        \hline
        Recommended Battery & 4-5 Cells @ 14.4-18.5V\\
        RPM & 670kv\\
        Max Current & 40A\\
        No Load Current & 10V/3.9A\\
        Current Capacity & 55A/15sec\\
        Internal Resistance & 0.04 ohm\\
        Weight & 303g\\
        Diameter of Shaft & 6mm\\
        Dimensions & 76x50mm\\
        \hline
        \end{tabular}
        \caption{Motor Parameters}
\end{figure}

\subsection{Motor Demands}
The boundary parameters mentioned previously are also essential in selecting
the components for the power stage of the motor controller. These parameters are
as follows;

\begin{figure}[h!]
        \centering
        \begin{tabular}{ll}
        \hline
        Parameter & Value \\
        \hline
        Average Speed & 30 km h$^{-1}$\\
        Peak Speed & 52 km h$^{-1}$\\
        Wheel diameter & 0.6m\\
        Wheel Reduction Ratio & 1:10\\
        Average RPM (for switching frequency) & 2653 RPM\\
        Peak RPM (for switching frequency) & 4598 RPM \\
        Peak Acceleration (for dv/dt FET capabilities) & 0.856 RPM s$^{-1}$\\
        \hline
        \end{tabular}
        \caption{Motor Boundaries}
\end{figure}
