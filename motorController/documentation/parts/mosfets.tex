%=== MOSFETS ===================================================================
\section{MOSFETs}
The selected MOSFET based on the motor parameters was the \href{http://www.electusdistribution.com.au/products_uploaded/ZT-2468.pdf}{IRF1405}

A summary of the MOSFETs parameters is below:

\begin{figure}[h!]
        \centering
        \footnotesize
        \begin{tabular}{llll}
        \hline
        & Parameter & Value & Conditions \\
        \hline
        $V_{(BR)DSS}$ & Drain-to-Source Breakdown Voltage & 55V & $V_{GS}=0V,~I_D=250\mu A$\\
        $I_{D}$ & Continuous Drain Current & 169A & $V_{GS}=10V,~T_C=25^{\circ}C$\\
        $I_{D}$ & Continuous Drain Current & 118A & $V_{GS}=10V,~T_C=100^{\circ}C$\\
        $I_{DM}$ & Pulsed Drain Current & 680A & \\
        $T_J$ & Operating Junction Temperature Range & $-55^{\circ}C - 175^{\circ}C$ & \\
        $Q_G$ & Total Gate Charge & $170nC - 260nC$ & $I_D=101A,~V_{DS}=44V,~V_{GS}=10V$\\
        $Q_{gs}$ & Gate-to-Source Charege & $44nC - 66nC$ & $I_D=101A,~V_{DS}=44V,~V_{GS}=10V$\\
        $Q_{gd}$ & Gate-to-Drain (Miller) Charge & $62nC - 93nC$ & $I_D=101A,~V_{DS}=44V,~V_{GS}=10V$\\
        $t_{r}$ & Rise time & 190ns & $V_{DD}=38V,~I_D=101A,~R_G=1.1\Omega,~V_{GS}=10V,$\\
        $t_{f}$ & Fall time & 110ns & $V_{DD}=38V,~I_D=101A,~R_G=1.1\Omega,~V_{GS}=10V,$\\
        \hline
        \end{tabular}
        \caption{IRF1405 Specifications}
\end{figure}

The H-bridge was chosen to be designed with two N-channel MOSFETs despite the
extra complexity involved in maintaining a sufficent gate voltage to the high
side N-channel MOSFETs, compared to the simplier P-channel MOSFETs. This is
because N-channel MOSFETs exhibit lower on resistances and hence minimized power
losses.

%-----------------------------------------------------------
\subsection{Switching Frequency}

%-----------------------------------------------------------
\subsection{Bootstrapping}
The N-Channel MOSFET proposed to be used on the high side of the power stage
requires a gate voltage above the source voltage by a minimum determined by
the gate threshold voltage, $V_{GS(Th)}$. In order to reach these voltages, the
simplest, and most widely used circuitry is the bootstrap charge circuit.
\footnote{\href{https://www.fairchildsemi.com/application-notes/AN/AN-9052.pdf}{Fairchild AN9052: Design Guide for Selection of Bootstrap Components}}

The necessary calculations are:
    \begin{equation}
    \Delta V_{boot} \leq V_{CC} - (V_{fbs}+V_{GS(min)} + V_{source})
    \end{equation}
    Where;
        \begin{itemize}
        \item $V_{CC}$ is the supply voltage to the gate driver
        \item $V_{fbs}$ is the static foward bias voltage across the bootstrap
        didode
        \item $V_{GS}$ is the minimum gate-to-source voltage required for the
        low on resistance across the MOSFET
        \item $V_{source}$ is the floating voltage of the source pin on the FET
        when connected to the load and driver
        \end{itemize}

    \begin{subequations}
    \begin{align}
    C_{boot} = \frac{Q_{total}}{\Delta V_{boot}}\\
    Q_{total} = Q_g + t_{on}\cdot(I_{LK} + I_{QBS}) + Q_{LS}\\
    I_{LK} = I_{LK,GS} + I_{LK,H} + I_{LK,D}\\
    \end{align}
    \end{subequations}
    Where;
        \begin{itemize}
        \item $Q_g$ is the gate charge on the FET
        \item $I_{QBS}$ is the operating current in the Gate Driver
        \item $t_{on}$ is the turning on interval of the FET
        \item $Q_{LS}$ is the level shift charge required per cycle
        \item $I_{LK}$ is the total leakage current
        \item $LK,GS$ indicates the gate leakage of the FET
        \item $LK,H$ indicates the high side floating supply leakage current
        \item $LK,D$ indicates the leakage current through the bootstap didoe
        \end{itemize}

%- - - - - - - - - - - - - - - - - - - - - - - - - - - - - - - - - - - -
\subsubsection{Bootstrap Didodes}
The bootstrap diode ($D_{BS}$) needs to be able to block the full power rail
voltage, which is seen when the high side MOSFET is switched
on. \footnote{\href{http://www.irf.com/technical-info/designtp/dt98-2.pdf}{IRF Design
Tip 98-02: Bootstrap Component Selection for Control IC's}}. The current rating
can be the product of total charge and switching frequency. Finally, certain
characteristics of the $D_{BS}$ is preferable; fast recovery time to minimize
leakage current (100nS or
less)\footnote{\href{http://www.ixysic.com/home/pdfs.nsf/www/AN-400.pdf/$file/AN-400.pdf}{IXYS
AN-400: IX2127 Design Considerations}}

Due to the low voltages
involved\footnote{\href{http://www.electronicspoint.com/threads/what-is-the-difference-between-schottky-and-fast-recovery-diode.97716/\#post-669477}{Community Forums, answer by John Popelish}}

%- - - - - - - - - - - - - - - - - - - - - - - - - - - - - - - - - - - -
\subsubsection{Level Shifters}

%- - - - - - - - - - - - - - - - - - - - - - - - - - - - - - - - - - - -
\subsubsection{Disadvantages}
Whilst bootstrapping has the advantage of being a simple and low cost
implementation, it is not without disadvantages. The duty-cycle and on-time is
limited by the requirement to refresh the charge in the bootstrap capacitor.

Negative voltage 2

%- - - - - - - - - - - - - - - - - - - - - - - - - - - - - - - - - - - -
\subsubsection{Latch Up on Source}
In the case of negative voltage in the
source\footnote{\href{https://www.fairchildsemi.com/application-notes/AN/AN-6076.pdf}{Fairchild
AN-6076: Design and Application Guide for Bootstrap Circuit for High Voltage
Gate Drive IC}}

